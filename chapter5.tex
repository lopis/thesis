%!TEX root = thesis.tex
\chapter{Work Plan}
\label{chap:plan}

\section*{}

\begin{figure}
	\caption{Work plan: Specification Phase}
	\label{dia:gantt}
	\begin{ganttchart}[
		x unit=0.004\textwidth,
		time slot format=little-endian
		]{1.2.2014}{30.6.2014}
		\gantttitlecalendar*{1.2.2014}{30.6.2014}{month=shortname} \\
		\ganttgroup{1. SOTA}{1.2.2014}{15.2.2014} \\

		\ganttgroup{2. Specification}{17.2.2014}{6.4.2014} \\
		\ganttbar{2.1 Tool Analysis}{17.2.2014}{2.3.2014} \\
		\ganttbar{2.2 Framework Analysis}{3.3.2014}{16.3.2014} \\
		\ganttbar{2.3 FIPA Lib Specification}{17.3.2014}{30.3.2014} \\
		\ganttbar{2.4 Software Specification}{17.3.2014}{30.3.2014} \\

		\ganttgroup{3. Development}{31.3.2014}{31.5.2014} \\
		\ganttbar{3.1 Code Gen. Tool}{31.3.2014}{18.5.2014} \\
		\ganttbar{3.2 Examples Creation}{7.4.2014}{21.4.2014} \\
		\ganttbar{3.3 Tests Creation}{7.4.2014}{21.4.2014} \\
		\ganttbar{3.4 Testing and Validation}{7.4.2014}{31.5.2014} \\

		\ganttgroup{4. Documentation}{5.5.2014}{29.6.2014} \\
		\ganttbar{4.1 Java Docs}{5.5.2014}{19.5.2014} \\
		\ganttbar{4.2 Sci. Paper}{5.5.2014}{31.5.2014} \\
		\ganttbar{4.3 Thesis}{5.5.2014}{30.6.2014} \\
	\end{ganttchart}
\end{figure}

The diagram in figure \ref{dia:gantt} makes an attempt to summarize all the tasks planned for the next months. The work plan can be grouped in three phases: specification phase, when the details of implementation will be defined with as much detail as possible, and the foundation for the documentation of the tool will be created; the development phase, when the actual programming work load will take place, including creation and execution of tests and finally, the writing of the thesis, which will continue beyond this five month period.

The \emph{specification phase} starts with an analysis of the tools to use (namely Eclipse, Spoon and ATL). The first two weeks of this phase will be dedicated to the study of both tools' API and documentation (2.1), as well as actual testing of its features, followed by two weeks dedicated to a thourough study of JADE and Repast and the mapping of equivalent features between them (2.2). Next comes the definition of the requisites for the FIPA-ACL library for Repast (2.3), as well as the specification for the rest of the code generation tool. (2.4).

The \emph{development phase} starts with a week exclusively dedicated to the creation of the general structure of the code generation tool, followed by the progressive development of all intended features (3.1). This task also includes the creation of a FIPA-ACL protocols library for Repast. Starting in the second week, sample projects will be created. These will be the test examples that the code conversion tool should be capable of converting (3.2). During this period, it is also expected that the creation of most unit tests should be completed, to ensure proper development of the tools (3.3). Validation should start as the tests creation is completed and as the development phase progresses (3.4).

Finally, in the \emph{documentation phase} three kinds of documents will be created. First, and essencially based on the in-code documentation already included in the development phase, Java Docs will be generated. This task will also contemplate the creation of a user manual for the code generation tool. The second document will be a scientific paper which is expectd to be submitted to an international conference or journal. Finally, this phase's main task is the writing of the thesis that will sum up all the details of the actual implementation and documentation of all results.