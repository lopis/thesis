%!TEX root = doc.tex
\section{FIPA in JADE} % (fold)
\label{sec:jade}

JADE is a Java framework for MAS development that complies with the wide range of FIPA specifications for agent interaction and management.

JADE uses the concept of behaviors which are adopted by agents that want to play a certain role or perform a task. Some of these behaviors are meant to be used in agent interaction. These interaction behaviors include two roles - the initiator and the responder. These two roles are specified in the FIPA interaction protocols. For instance, Figure \ref{fig:fipa_request} represents the interactions between two agents - the initiator and the participant or responder - in the FIPA Request protocol.

\begin{figure}[h]
	\centering
	\includegraphics[width=2.5in]{figures/fipa_request.png}
	\caption{FIPA Request Interaction Protocol}
	\label{fig:fipa_request}
\end{figure}

In order to create an application using these protocols, programers only need to extend these behaviors and implement their own handlers. All the complexity regarding the interaction and networking infrastructure is hidden and taken care of by JADE.

FIPA interaction protocols are not the only FIPA standars present in JADE. The Agent Comunication Language (ACL) is also present, as well as the ontology service, agent management service and the directory facilitator service.
This architecture allows the programmer to focus on the implementation of agent behavior without having to deal with the details of the protocols and the development of the network infrastructure. \cite{bellifemine2003jade}