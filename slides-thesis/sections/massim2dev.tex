%!TEX root = ../thesis.tex

%%%%%%%%%%%%%%%%%%%%%%%%%%%%%%%%%%%%%%%%%%%%%%%%%%%%%%
%%%%%%%%%%%%%%%%%%%%%%%%%%%%%%%%%%%%%%%%%%%%%%%%%%%%%%
\subsection{MASSim2Dev}
\begin{frame}{MASSim2Dev - Ferramenta de Conversão}

	\begin{itemize}
	\item MASSim2Dev (\emph{MAS Simulation to Development code conversion tool})
	\item Estabelece a ponte entre o desenvolvimento e simulação de MAS utilizando o SAJaS
	\end{itemize}
\end{frame}

\subsection{Transformações de código}
\begin{frame}{MASSim2Dev - Ferramenta de Conversão}

	Transformações de código Java: ferramentas estudadas

	\begin{itemize}
	\item ATL (``ATLAS Transformation Language'') - transformações de modelos através de AST usando linguagem específica;
	\item Spoon - transformações de código Java baseadas em anotações, usando Java
	\item \color{FEUPCor} JDT (``Eclipse Java Development Tools'') - criação de plugins para Eclipse, que permitem edições de alto nível, assim como via AST

	\end{itemize}
\end{frame}

\subsection{Processo de conversão}
\begin{frame}{MASSim2Dev - Ferramenta de Conversão}

	Após a conversão, não restam dependências da plataforma original.
	\begin{figure}[h]
		\centering
		\includegraphics[height=5cm]{../figures/conversion_representation.pdf}
	\end{figure}
\end{frame}

\begin{frame}{MASSim2Dev - Ferramenta de Conversão}

	Algoritmo
	\begin{samepage}
		\begin{enumerate}
		  \item Clonar o projeto selecionado
		  \item Mudar as referências a \emph{imports} em cada classe
		  \item Introduzir as bibliotecas necessárias (JADE, Repast...) e adiciona-las ao \emph{build path} do projeto
		  \item Reparar hierarquia (p.e. classes que estendem \texttt{RepastAgent} passam a estender \texttt{Agent})
		\end{enumerate}
	\end{samepage}
\end{frame}

