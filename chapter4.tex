%!TEX root = thesis.tex
\chapter{Validation Perspectives} \label{chap:validation}

\section*{}

This chapter makes a brief description of the expected development environment, including the tools that will be used, followed by an overview of the perspectives of validation for the results of this thesis.

\section{Setup}

This thesis will make use of JADE 4.3.1 (6 December 2013) and Repast Symphony 2.1 (12 August 2013), which are the latest stable versions of these frameworks as of the writing of this report. Repast has implementations in languages other than Java but, since JADE is written in this language, it is only natural to use the Java implementation of Repast for the sake of simplifying the conversion process. The latest stable version of the Java platform, Java 7, is compatible with both frameworks and will be used in the development phase of this thesis.

\section{Validation and Evaluation}

Because the validation of this tool's ability to correctly generate code depends entirely on the functionality of said code, unity testing can be used in order to verify conformity of a series of examples. For testing purposes four sets of unit testing will be developed

\begin{description}
  \item[The first] set of tests will cover the code of the code generation tool itself;
  \item[The second] one will cover the code of the Repast/FIPA-ACL implementation library;
  \item[The third] will include a group of example Repast code to be converted to JADE - the unit tests will be applied on the generated code;
  \item[The fourth] will perform the complementary tests to the third set, testing the conversion of JADE code to Repast.
\end{description}

While unit testing allows to verify the generated code functionality, it's also important to actually evaluate the quality of the generated MAS or MABS. The agents generated by the code conversion tool should display similar behavior to the original agent's.
