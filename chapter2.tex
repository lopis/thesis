%!TEX root = thesis.tex
\chapter{Revisão Bibliográfica} \label{chap:sota}

\section*{}

% Neste capítulo é descrito o estado da arte e são apresentados trabalhos
% relacionados para mostrar o que existe no mesmo domínio e quais os problemas
% em aberto. Deve deixar claro que existe uma oportunidade de desenvolvimento
% que cobre alguma falha concreta .

% O capítulo deve também efetuar uma revisão tecnológica às principais
% ferramentas utilizáveis no âmbito do projeto, justificando futuras escolhas.

In this chapter I will summarize and analyze some works related to the use of Repast and JADE frameworks in the development of MAS. The related
work is sorted in four categories.

First, I discuss of some work done on integrating both JADE and Repast simultaneously by developing a middleware framework.  While the goal of this thesis is not to use both frameworks simultaneously, these works give a valuable insight into the shortcomings of both frameworks as well as providing some interesting comparisons between their features.

Second, I make an overview of the FIPA standards for agent communication protocols along with a discussion of implementations of these protocols in Repast.

Then, I review some existing Java code generation and transformation tools. These tools will aid me in the automation of code refactoring, detection of code patterns and library calls.

Finally, I analyze some works on code generation and transformation techniques.
More than just tools, there are proper methodologies of automating code generation.

\section{JADE and Repast integration}

\subsection{
	JRep - Extending Repast Symphony for JADE Agent Behavior Components}
\begin{quote}
	``JREP [is] a novel integration of JADE and Repast Symphony that efficiently combines the macro and micro perspective with an interaction layer. It allows to see not only the overall system behavior, but also the individual together with its interests, goals and the communication to
	others for local coordination and cooperation. Scheduling of the agents, (time) synchronization and the registration of new agents with the environment has been solved.'' \cite{gormer2011jrep}
\end{quote}

\subsection{
	MISIA: Middleware Infrastructure to Simulate Intelligent Agents}
\begin{quote}
	``MISIA allows simulation, visualization and analysis of the agent’s behavior. MISIA makes use of technologies for the development of multi-agent systems known and widely used, and combines them so that it is possible to use their capabilities to build highly complex and dynamic systems. On one hand, it is JADE, the most widely used platform for based software agents middleware. On the other hand, it is Repast, a free and open-source agent-based modeling and simulation toolkit.'' \cite{gormer2011jrep}
\end{quote}

% end % JADE and Repast integration

\section{FIPA standards in Repast}

% end % FIPA standards in Repast

\section{Code generation tools}

% end % Code generation tools

\section{Code generation techniques}

% end % Code generation techniques



\section{Summary}


