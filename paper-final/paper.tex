\documentclass[a4paper,twoside]{article}

\usepackage{epsfig}
\usepackage{subfigure}
\usepackage{calc}
\usepackage{amssymb}
\usepackage{amstext}
\usepackage{amsmath}
\usepackage{amsthm}
\usepackage{multicol}
\usepackage{pslatex}
\usepackage{apalike}
\usepackage{SCITEPRESS}
\usepackage[small]{caption}

\subfigtopskip=0pt
\subfigcapskip=0pt
\subfigbottomskip=0pt

\begin{document}

\title{From simulation to development in MAS: A JADE-based Approach}

\author{\authorname{João P. C. Lopes\sup{1} and Henrique Lopes Cardoso\sup{1}}
\affiliation{\sup{1}LIACC, Faculty of Engineering of University of Porto, Rua Dr. Roberto Frias, Porto, Portugal}
\email{\{lopes.joao.pedro, hlc}@fe.up.pt}
}

\keywords{MAS, MABS, Framework Integration}

\abstract{Multi-agent systems (MAS) present an interesting approach to the efficient development of modular systems. Frameworks exist that aid the development of this class of systems but, most of the time, no single one serves a project fully. Some works propose the integration of frameworks with other frameworks or with custom adapters to complement most of their missing features. This paper proposes the use of JADE as a MAS development environment and the use of simulation frameworks like Repast as the engine for developing JADE-based simulations. Repast is extended with and API called SAJaS, which implements a set of JADE features and allows the creation of simulations using a complex agent infrastructure. A conversion tool called MASSim2Dev bridges the gap between MAS development and simulation by mapping SAJaS's and JADE's API. This solution provides increased simulation performance while enabling programmers to quickly get started with development or simulation of their multi-agent models. Validation tests show how using MASSim2Dev preserves the original functionality of the system.}

\onecolumn \maketitle \normalsize \vfill

\section{\uppercase{Introduction}}
\label{sec:introduction}

\noindent 




\section{\uppercase{Conclusions}}
\label{sec:conclusion}

\noindent 


\vfill
\bibliographystyle{apalike}
{\small
\bibliography{myrefs}}

\vfill
\end{document}

