%-----------------------------------------------
% Template para criação de resumos de projectos/dissertação
% jlopes AT fe.up.pt,   Fri Jul  3 11:08:59 2009
%-----------------------------------------------

\documentclass[9pt,a4paper]{extarticle}

%% English version: comment first, uncomment second
%\usepackage[portuguese]{babel}  % Portuguese
%\usepackage[english]{babel}     % English
\usepackage{graphicx}           % images .png or .pdf w/ pdflatex OR .eps w/ latex
\usepackage[utf8]{inputenc}     % 8 bits using UTF-8
\usepackage{url}                % URLs
\usepackage{multicol}           % twocolumn, etc
\usepackage{float}              % improve figures & tables floating
\usepackage[tableposition=top]{caption} % captions
%% English version: comment first (maybe)
\usepackage{indentfirst}        % portuguese standard for paragraphs
%\usepackage{parskip}

%% page layout
\usepackage[a4paper,margin=30mm,noheadfoot]{geometry}

% For subfigure
\usepackage{subcaption}

%% space between columns
\columnsep 12mm

%% headers & footers
\pagestyle{empty}

%% figure & table caption
\captionsetup{figurename=Fig.,tablename=Tab.,labelsep=endash,font=bf,skip=.5\baselineskip}

%% heading
\makeatletter
\renewcommand*{\@seccntformat}[1]{%
  \csname the#1\endcsname.\quad
}
\makeatother

%% avoid widows and orphans
\clubpenalty=300
\widowpenalty=300

\begin{document}

\title{\vspace*{-8mm}\textbf{\textsc{
  From simulation to development in MAS:\\
  A JADE-based Approach
}}}
\author{
  \emph{João P. C. Lopes}\\[2mm]
  \small{Dissertação desenvolvida sob orientação de \emph{Henrique Lopes Cardoso}}
}
\date{23 de Junho 2014}
\maketitle
%no page number 
\thispagestyle{empty}

\vspace*{-4mm}\noindent\rule{\textwidth}{0.4pt}\vspace*{4mm}

\begin{multicols}{2}

%!TEX root = thesis.tex
\section{Resumo}

Os sistemas multi-agente (MAS, \emph{Multi-Agent Systems}) exprimem uma abordagem interessante no desenvolvimento de sistemas modulares e eficientes, o compostos por elementos computacionais autónomos -- chamados agentes. Existem ferramentas de software que facilitam o desenvolvimento desta classe de sistemas que podem variar entre ferramentas de âmbito geral até ferramentas focadas num domínio específico.

\subsection{Motivation}
Simulações baseadas em multi-agentes (MABS) são usadas no ciclo de desenvolvimento de MAS completos -- por exemplo para realização de testes -- pela performance superior das simulações. No entanto, as plataformas de desenvolvimento de MAS são geralmente pouco apropriadas para desenvolver MABS por serem pouco escaláveis \cite{mengistu2008scalability}. Trabalhos estudados revelam a existência de interesse em integrar MABS no ciclo de desenvolvimento de MABS. Por fim, existe a possibilidade de automatizar parcialmente o desenvolvimento de MAS a partir de simulações previamente testadas.

O JADE \cite{bellifemine2007developing} é uma framework para desenvolvimento de MAS distribuídos que implementa as especificações da FIPA. Quando o número de agentes é elevado, a sua arquitetura multi thread afeta negativamente a performance de simulações baseadas no JADE.

O Repast \cite{collier2003repast} é uma ferramenta de criação de MABS rica em GUIs e estatísticas em tempo real. Consegue facilmente gerir grandes números de agentes numa única simulação. Ao contrário do JADE, não dispõem de uma infraestrutura de interação e criação de agentes.

A motivação desta dissertação reside no potencial ganho de performance ao usar frameworks de MABS para simular MAS mais complexos que aqueles tipicamente criados usando essas frameworks. Alguns trabalhos \cite{garcia2011misia,gormer2011jrep} propõem a integração de ferramentas de simulação no JADE, quer criando uma camada de simulação por cima deste, ou integrando-o com outras plataformas como o Repast.

\subsection{Objetivos}
Nesta dissertação é proposta uma solução integrada para integrar o desenvolvimento e simulação de MAS. Para tal, dois sub-objetivos foram identificados.

\begin{enumerate}
  \item \textbf{Primeiro}, a criação de um adaptador ou API que permite que os programadores se abstraiam da ferramenta de simulação. Isto é possível reimplementando funcionalidades do JADE, incluindo especificações FIPA sobre gestão de e interação entre agentes. Uma arquitetura idêntica à do JADE permite uma conversão mais direta do código.

  \item \textbf{Segundo}, o desenvolvimento de um mecanismo de conversão de código. Ao permitir a abstração da plataforma de simulação, é possível a criação de uma ferramente que realize a conversão direta de um MABS num MAS equivalente.
\end{enumerate}

\subsection{SAJaS}

O SAJaS (\emph{Simple API for JADE-based Simulations}) implementa um conjunto de funcionalidades baseadas no JADE; foram reimplementadas de raiz de modo a simplificar algumas características internas e a aumentar a performance da simulação, preservando o comportamento do MAS. Entre as funcionalidades implementadas, destacam-se as especificações para gestão de agentes, serviço de mensagens e interação entre agentes da FIPA.

Um dos desafios ao desenvolver o SAJaS foi a adaptação da execução assíncrona do JADE ao ambiente síncrono baseado em ``ticks'' do Repast. Exceto quando ocorrem durante a iniciação e terminação dos agentes no JADE, as suas ações estão tipicamente encapsuladas em ``Behaviours'' que executam em concorrência ou em paralelo. No SAJaS, os behaviours são executados sequencialmente. Para emular a comunicação assíncrona do JADE no SAJaS, as mensagens são mantidas na fila de espera do destinatário até serem necessárias e processadas -- ao contrário do que se verifica no JADE, em que as mensagens despoletam o ``Behaviour'' apropriado de imediato.

Os ``behaviours'' no SAJaS seguem o mesmo ciclo de vida que no JADE, incluindo os métodos \texttt{action}, \texttt{onStart}, \texttt{done} e \texttt{onEnd}. Também está presente a implementação de protocolos de interação, nomeadamente o ``FIPA Request'' e o ``FIPA Contract Net''.

Foi feito um esforço consciente para manter o SAJaS genérico no que diz respeito a dependências ao Repast, abrindo as portas à integração futura em outras ferramentas de simulação sem a necessidade de modificar a API.

\subsection{MASSim2Dev}

O MASSim2Dev (\emph{MAS Simulation to Development code conversion tool}) é uma ferramenta de que, utilizando o SAJaS, estabelece a ponte entre o desenvolvimento e simulação de MAS.

O JDT (www.eclipse.org/jdt/) foi selecionado para desenvolver esta ferramenta. Entre as suas capacidades, destacam-se a capacidade de (dinamicamente) clonar projetos Java, gerir classes, ``imports'', métodos e variáveis como objetos e de realizar manipulações de código complexas sem analisar todo o código (``parsing''). Também permite a utilização de uma árvore sintática (AST) de alto nível para manipulação direta do código fonte.

Após a conversão do código, não restam dependências à plataforma anterior no projeto gerado. Naturalmente, isto apenas se verifica para as funcionalidades comuns ao JADE e ao SAJaS presentemente.

\subsection{Validação}
Foram criados três testes para demonstrar que o comportamento das funcionalidades baseadas no JADE que foram reimplementadas no SAJaS é equivalente ao das correspondentes no JADE. Além disso, foi possível obter ganhos na performance de simulações baseadas no SAJaS.

O primeiro exemplo consiste numa ``Contract Net'' entre um comprado e múltiplos vendedores. No segundo exemplo, múltiplas ``Contract Nets'' correm concorrentemente e alguns dos compradores utilizam valores de confiança para estabelecer contratos. No terceiro exemplo, um jogo de tabuleiro chamado Risk, desenvolvido em JADE antes da conceção desta dissertação foi sujeito à conversão do seu código. O objetivo deste último era testar o sistema num exemplo real que não tivesse sido desenvolvido de propósito para esta dissertação.

Estes cenários cobrem todas as funcionalidades do SAJaS. Foi possível demonstrar que é possível traçar a ponte entre o JADE e o Repast e que os ganhos de performance na simulação são significativos.

\subsection{Conclusions}
O estudo da bibliografia demonstrou que existe um interesse em desenvolver ferramentas para simulação de MAS devido a limitações existentes em plataformas de desenvolvimento.

A solução proposta, composta pelo SAJaS e pelo MASSim2Dev, permite o desenvolvimento de simulações usando funcionalidades baseadas no JADE numa plataforma de simulação como o Reapast.

Neste momento, um subconjunto das funcionalidades do JADE estão presentes no SAJaS. No entanto, os testes de validação mostram que é já possível criar MAS com alguma complexidade e que a sua conversão com o MASSim2Dev preserva o comportamento original.

Algum trabalho futuro sugerido inclui expandir o conjunto de funcionalidades do JADE presentes no SAJaS, extender o suporte nativo da API a outras plataformas de simulação, complementar o plugin com configurações do utilizador e permitir a criação de gráficos e visualizações através do SAJaS.

%%English version: comment first, uncomment second
\bibliographystyle{unsrt-pt}  % numeric, unsorted refs
%\bibliographystyle{unsrt}  % numeric, unsorted refs
\bibliography{myrefs}

\end{multicols}

\end{document}
