%!TEX root = thesis.tex
\chapter*{Abstract}

Multi-agent systems (MAS) present an interesting approach to the efficient development of modular systems. MAS are composed by autonomous computational units called agents that are programmed to \emph{compete} or \emph{work together}, for instance in order to solve computational problems, engage in automatic negotiation or play computer games. Frameworks exist that aid the development of this class of systems and they range from mostly general-purpose frameworks to domain-specific in an array of different domains.

Multi-agent-based simulations (MABS) are sometimes used on the course of development of a full-featured MAS - for instance, for testing purposes. However, most platforms for MAS development are not well suited for MABS development \cite{mengistu2008scalability}.

JADE \cite{bellifemine2003jade}, a very popular MAS development framework allows the creation of seamless distributed agent systems and complies with FIPA standards for agent interaction. Unfortunately, its multi-threaded architecture falls short in delivering the necessary performance to run a local simulation with a large number of agents. Repast \cite{collier2003repast} is an agent-based simulation toolkit that allows creating simulations using rich GUI elements and real time agent statistics. Unlike JADE, though, Repast lacks much of the infrastructure for agent creation and interaction.

Some authors \cite{garcia2011misia,gormer2011jrep}, as reviewed in this report, proposed an integration of JADE and Repast by means of a middleware, essentially representing the agent in both and taking advantage of the features provided by the frameworks.

In this thesis, a code generation tool is proposed, capable of not only generating a Repast simulation from an existing JADE MAS, but also of creating a full featured JADE application from a Repast-based simulation. An implementation of FIPA's interaction protocols will be proposed for Repast as a means to deliver the mentioned conversion tool. The most immediate advantage over the previous approaches is that existing systems can make use of this tool to create appropriate simulations where agent interaction can be more easily analyzed. Furthermore, proficient programmers in one framework can quickly get started in the development for the complementary framework.

\chapter*{Resumo}

Os sistemas multi-agente (SMA) exprimem uma abordagem interessante no desenvolvimento de sistemas modulares e eficientes. Os MAS são compostos por elementos computacionais autónomos - chamados agentes - que são programados para \emph{competir} ou \emph{colaborar} de modo a, por exemplo, resolver problemas computacionais, iniciar negociação automática ou participar em jogos de computador. Existem ferramentas de software que facilitam o desenvolvimento desta classe de sistemas que podem variar entre ferramentas de âmbito geral até ferramentas focadas num domínio específico.

As simulações baseadas em agente (SBA) são frequentemente utilizadas durante o desenvolvimento de MAS completos - por exemplo, afim de testar o sistema. No entanto, a maior parte das plataformas para desenvolvimento de SMA não são apropriadas para a criação de SBA \cite{mengistu2008scalability}.

O JADE \cite{bellifemine2003jade} é um exemplo de uma plataforma de desenvolvimento de SMA que permite a criação de sistemas distribuídos de agentes de forma simplificada cumprindo, ainda, os standards da FIPA (\emph{Foundation for Intelligent Physical Agents}) sobre protocolos de interação entre agentes. No entanto, a sua arquitetura em \emph{multi-thread} não garante a performance necessária para a execução de simulações com um elevado número de agentes. O Repast \cite{collier2003repast} é uma plataforma de criação de simulações baseadas em agentes que permite criar simulações ricas em interfaces gráficas para visualização de dados históricos e em tempo real sobre os agentes. Ao contrário do JADE, o Repast não dispõe de uma infraestrutura para criação de agentes e interação entre eles.

Alguns autores estudados neste relatório \cite{garcia2011misia,gormer2011jrep} propõem uma integração entre o JADE e o Repast por meio de uma camada de software intermédio, fazendo representar cada agente duplamente - uma vez em cada plataforma, essencialmente tirando partido das capacidades de ambas as plataformas em simultâneo.

Para esta dissertação é proposta uma ferramenta de geração automática de código capaz não só de gerar uma simulação baseada em Repast a partir de um SMA baseado em JADE, mas também de criar um SMA em JADE partindo de uma simulação baseada em Repast. Será também criada uma proposta de implementação, para a plataforma Repast, dos standards da de interação entre agentes da FIPA, tornando possível a conversão de código entre as duas plataformas. A vantagem mais imediata sobre a proposta anterior é que SMA já existentes podem tirar partido da ferramenta proposta para rapidamente criar uma simulação com funcionamento equivalente e ter acesso a ferramentas de visualização e estatísticas do sobre o comportamento dos agentes. Adicionalmente, programadores proficientes numa das plataformas pode iniciar rapidamente o desenvolvimento da plataforma complementar.

