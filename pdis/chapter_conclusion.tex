%!TEX root = thesis.tex
\chapter{Conclusions} \label{chap:concl}

\section*{}

Existing works have proposed ways of simplifying the use of simulations during the development of MAS. However, for this thesis, a new methodology is proposed, distinct from those proposed in the available literature.

\section{Context Summary}
Two related works were selected for analysis in this report: JREP and MISIA. To approach the need for creating MABS in JADE, these authors proposed the integration of both JADE and Repast in the same platform by means of a middleware.This approach effectively brings to the development with JADE such simulation tools as execution control GUI and a panoply of charts that display both simulation records as well as real-time information about the agents. The problem with this approach is the need of using this integrated framework from the beginning of development and even after deployment of the MAS, i.e. the MAS is dependent on both JADE and Repast.

\section{Proposal Overview}
The tool proposed in this report will allow developers of MAS in JADE to quickly generate an equivalent simulation in Repast. It will also allow a MAS to be initially designed as a MABS in Repast and converted later on.

On the course of this thesis, and as defined in the work plan (chapter \ref{chap:plan}), the details of the methodology that will be used to generate the code will be specified. This methodology will comprise ways of detecting patterns in the code, their use of certain objects, classes, structures and library resources and convert them to their equivalent in the complementary framework. Three software solutions were selected as potential tools to use in the code conversion tool to be developed: JDT, ATL and Spoon. These tools will be studied to determine which provide the most appropriate group of features. Some of the main possibilities presented by these tools are the introspection of code, detection of code patterns, code reflection and refactoring, model transformations and AST transformations.





%\section{Trabalho Futuro}


