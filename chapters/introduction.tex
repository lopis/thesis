%!TEX root = ../thesis.tex
\chapter{Introduction}
\label{chap:introduction}
%<In this paragraph present the context of the thesis, introducing the subject of MAS, MABS and some of their uses. Refer that standards exist and why. Introduce the next sections too.>
Multi-Agent Systems (MAS) are composed of autonomous computational elements capable of interacting with each other, called agents. The development of this class of systems comprises an interesting software paradigm but in terms of computer science history, MAS are a recent subject, having gained significant traction only after the mid 1990's \cite{wooldridge2008introduction}. With multiple applications such as problem solving, simulation, trading, negotiation, computer games and logistics using an efficient and modular development approach, MAS enjoyed a rapid growth in popularity and are in widespread use nowadays \cite{ferber1999multi}.

Presently, tools and frameworks for the development of all sorts of MAS are as diverse as uses exist for them. This thesis describes a concrete problem of integrating frameworks from different domains.

\section{Problem}
%The following problems will be explained:

% - There is no universal standard. Most systems don't use any standards
Although their use is certainly widespread, there is no universal general purpose standard for MAS development, since each system has different needs. Many times, such systems are created from scratch, meaning that the developers must define all features of the system - such as its agents, their behaviour, communication and organization, using conventional programming languages and tools. However, several frameworks exist that offer some level of abstraction from the code, allowing for a more conceptual approach to
MAS development \cite{gormer2011jrep}. 

% - Full featured MAS development frameworks are often not the most appropriate to develop simulations for their complex architecture
Most uses of MAS, for instance in negotiation, games or logistics, demand a small number of agents, typically with larger resource demands but without any need for global control of execution, i.e. it is perfectly reasonable for these types of systems to be based on events and for its agents to work asynchronously. In contrast, Multi-Agent-Based Simulations (MABS) are usually implemented using a large number of lightweight agents with a small resource footprint. MAS development frameworks generally provide the programmer with a range of features such as execution control, communication protocols or agent awareness capabilities. In spite of that, most frameworks that focus on MAS development lack synchronization mechanisms and lightweight agent infrastructure required by MABS. One of the main goals of simulations is to be able to visualize real-time, as well as historical data that allow to study emergent and evolutionary phenomena. \cite{mengistu2008scalability}

% - Porting code from one framework to the other is typically not a feasible solution 
When an application has been developed using a MAS development framework and a need later arises for the creation of simulations, porting the source code to an appropriate MABS development framework is a labour-intensive task since not only the syntax and API of the new framework is significantly different, but conceptually speaking, the adaptation may require significant changes to the application.


\section{Motivation}
% It is useful that MAS be tested in a controlled simulation environment using proper simulation tools that may not be available in their destination platform;
Interest exists in the simulation of MAS. At any point of the development of the system, it may be valuable to test MAS in a local and controlled simulation environment and to take advantage of some features present in MABS frameworks that are not available in the destination platform of the system.The rationale for the creation of simulated agent systems is usually concerned with simulation performance. Simulations typically have a higher performance than complex MAS frameworks. For many popular MAS frameworks, there is an opportunity to gain performance when executing tests and simulations. 

% JADE and Repast are popular tools in widespread use and are well documented and supported by their communities so it is easier to build on top of them;
% "It is feasible to bridge the gap between MAS simulation and development by embedding FIPA-standards and JADE features in a simulation framework."
% "The development of a robust MAS can be partially automated from a previously tested simulation."
As some works suggest \cite{gormer2011jrep,garcia2011misia,warden2010towards}, it is feasible to bridge the gap between MAS simulation and development. For instance, JADE and Repast are popular tools in widespread use and are well documented and supported by their communities so it is easier to build on top of them. It it possible to establish this bridge by embedding FIPA-standards and JADE features in Repast.

\section{Goals}
The main goal of this thesis was to develop a code conversion tool. In order to bring MAS development and simulation together, this tool would allow to convert the code written for a development tool into code for a simulation tool. 

JADE and Repast were chosen over other platforms mainly for their popularity and widespread use - not dismissing their quality, of course. As an example of an alternative framework, Cougaar (Cognitive Agent Architecture)\cite{helsinger2004cougaar} proposes solving the problem explained above with a fully featured agent architecture, while maintaining high performance and scalability required for simulation. It doesn't, however, implement any interaction standards; messages are exchanged by means of serialized Java objects. Therefore, another goal was to develop an API that replicates the essential JADE features, while abdicating of JADE's networking infrastructure and more complex internal features in favour of gaining in simulation performance. This includes enabling interaction standards in simulation frameworks. While Repast is the featured framework for simulation development, it was part of this goal to develop an API that was sufficiently generic to allow for future enhancements and support for new platforms.

Programmers that wish to use the code conversion tool should not be forced to introduce significant changes to the original code in order to be able to use the tool. The goal was that the tool should be capable of converting the code \emph{as-is} and generate working models. The generated code must also preserve the functionality of the original code - meaning that the re-conversion must generate code that is equivalent to the original one.

% To enable interaction standards in simulation frameworks;
% To develop an API that replicates the essential JADE features, while abdicating of JADE's networking   infrastructure and more complex internal features in order to gain in simulation performance;
% To develop an automatic code conversion tool (CCT) that uses this API to transform Repast-based simulations into JADE MAS;
% To add the possibility to convert some JADE MAS into simulations, based on the API and using the code conversion tool;
% To validate generated code by confirming that the execution of the simulation and the generated MAS produce identical behaviour;
% To make the API generic enough, allowing for future extension to support multiple simulation tools

\section{Contents}

This thesis documents the development of a tool that converts a MABS created in Repast into MAS that uses JADE. Conversely, it should allow the conversion of a JADE MAS into a Repast MABS as well. This tool is useful in the context of development of a MAS whose development started as a MABS or when the need to create a simulation arises during development. JADE and Repast were chosen for this thesis not only for their quality but also for their widespread use, available source code and documentation.

Chapter \ref{chap:background} starts by surveying tools whose goal is to produce MABS. Three frameworks were selected for a more detailed study because their approach is the most relevant to the goals of this thesis. They propose solutions based on enhancing JADE to enable simulations capabilities in it. The rest of the chapter is dedicated to comparing JADE and Repast's features and to include an introduction to some concepts regarding FIPA specifications.

Chapter \ref{chap:solution} provides a conceptual definition of the developed tools, including an overview of their features and usage scenarios. This chapter also describes how FIPA Specifications are present in the API. To better understand how the code conversion tool, some background study is presented before describing the features of tool.

Chapter \ref{chap:architecture} gives a detailed description of the software architecture, including a description of how agents execute internally, in the API. This chapter is concluded with a discussion on the perspectives for extending the tools, teasing for the discussion of future work in the conclusions.

Chapter \ref{chap:validation} presents scenarios used to validate the system. They were subjected to code conversion to verify that the correct execution of the code had been preserved. Their performance was also subject to analysis.

This thesis is concluded with a description of suggested future work and some final notes and conclusions.