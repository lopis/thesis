%!TEX root = ../thesis.tex
\chapter{Introduction}
\label{chap:introduction}
%<In this paragraph present the context of the thesis, introducing the subject of MAS, MABS and some of their uses. Refer that standards exist and why. Introduce the next sections too.>
Multi-Agent Systems (MAS) are composed of autonomous computational elements capable of interacting with each other, called agents. The development of this class of systems comprises an interesting software paradigm but in terms of computer science history, agent-oriented programming is a relatively new subject, having gained significant traction only after the mid 1990's \cite{wooldridge2008introduction}. Today, the world of MAS development is fairly fragmented and several tools and frameworks exist, each fitting the needs of a subset of all developers. More than creating newer, better frameworks, it is interesting to survey what was created an integrate some of the available works.

\section{Context}

The focus of MAS is in the interaction between agents which are typically simple in their architecture but capable of being autonomous in their decisions. When working with groups of agents, it is possible to create complex applications even if each agent is not ``aware'' of the greater result of its actions. One can create a parallelism between MAS and our bodies: our cells, autonomous and simple beings, exhibit a much more complex behaviour when working together. A cell in out skin, though, is not aware of what is going on in some other organ, but the body still functions as a whole\cite{ferber1999multi}.

MAS enjoyed a rapid growth in popularity and are in widespread use nowadays. Some examples of their uses are simulating urban traffic or social environments, solving complex non-deterministic problems, electronic trading and negotiation, computer games and logistics. This list is far from exhaustive and is only meant to illustrate the versatility of MAS.

Presently, tools and frameworks for the development of all sorts of MAS are as diverse as uses exist for them. MAS development frameworks offer programmers an abstraction layer from software specification, allowing them to think more conceptually about agent-based applications. Some features found in many of these frameworks are agent architectures that simplify the creation of new agents, messaging services that provide simple interaction between them, networking infrastructure to allow communication between agents in different hosts and registries that index agents and facilitate searching for other agents\cite{survey2}.

\section{Problem}
%The following problems will be explained:

% - There is no universal standard. Most systems don't use any standards
Although their use is certainly widespread, there is no general purpose standard for MAS development, since each system has different needs. Many times, these systems are created from scratch, meaning that the developers must define all features of the system - such as its agents, their behaviour, communication and organization, using conventional programming languages and tools. Even though many frameworks offer some level of abstraction from the code, often no single framework satisfies a project's needs.

% - Full featured MAS development frameworks are often not the most appropriate to develop simulations for their complex architecture
Most uses of MAS, for instance in negotiation, games or logistics, demand a small number of agents, typically with larger resource demands but without any need for global control of execution -- it is acceptable for these types of systems to be based on events and for its agents to work asynchronously. In contrast, Multi-Agent-Based Simulations (MABS) are usually implemented using a large number of lightweight agents with a small resource footprint. Taking the example of traffic simulation, these applications can have many dozens or hundreds of agents representing vehicles, pedestrians and traffic lights. Furthermore, simulations usually feature global synchronization mechanisms.

MAS development frameworks generally provide the programmer with a range of features such as execution control, communication protocols or agent awareness capabilities. In spite of that, most frameworks that focus on MAS development lack synchronization mechanisms and lightweight agent infrastructure required by MABS. One of the main goals of simulations is to be able to visualize real-time, as well as historical data that allow to study emergent and evolutionary phenomena. \cite{mengistu2008scalability}

% - Porting code from one framework to the other is typically not a feasible solution 
When an application has been developed using a MAS development framework and a need later arises for the creation of simulations, porting the source code to an appropriate MABS development framework is a labour-intensive task since not only the syntax and API of the new framework is significantly different, but conceptually speaking, the adaptation may require significant changes to the application.


\section{Motivation}
% It is useful that MAS be tested in a controlled simulation environment using proper simulation tools that may not be available in their destination platform;
Interest exists in the simulation of MAS. At any point of the development of the system, it may be valuable to test MAS in a local and controlled simulation environment and to take advantage of some features present in MABS frameworks that are not available in the destination platform of the system.The rationale for the creation of simulated agent systems is usually concerned with simulation performance. Simulations typically have a higher performance than complex MAS frameworks. For many popular MAS frameworks, there is an opportunity to gain performance when executing tests and simulations. 

% JADE and Repast are popular tools in widespread use and are well documented and supported by their communities so it is easier to build on top of them;
% "It is feasible to bridge the gap between MAS simulation and development by embedding FIPA-standards and JADE features in a simulation framework."
% "The development of a robust MAS can be partially automated from a previously tested simulation."
As some works suggest \cite{gormer2011jrep,garcia2011misia,warden2010towards}, it is feasible to bridge the gap between MAS simulation and development. For instance, JADE and Repast are popular tools in widespread use and are well documented and supported by their communities so it is easier to build on top of them. The main motivation for this work is thus the potential gains in establishing this bridge by embedding FIPA-standards and other JADE specific features into a simulation tool like Repast. The development of robust MAS can be partially automated from a previously tested simulation.

\section{Goals}
The main goal of this thesis is to develop a solution for bridging the domains of simulation and MAS. In order to do that, two main sub-goals were identified.

\begin{enumerate}
	\item \textbf{First}, the creation of an adapter or API that would allow developers to abstract from simulation frameworks' features and use familiar ones present in MAS development frameworks, thus creating ``MAS-like MABS''. This approach allows the resulting code to be easily ported to a full featured MAS framework.
	\item \textbf{Second}, the development of a code conversion mechanism. By abstracting from the simulation tools and creating a MAS-like MABS, it becomes possible and more straightforward to engineer a tool that performs the automatic conversion of these MABS into equivalent MAS.
\end{enumerate}

JADE and Repast were chosen over other frameworks for multiple reasons. Both are very popular an in widespread use. In consequence, extensive documentation and many examples and applications created by the community are available for use and study. Furthermore, their source is free and open, which enables the development of other tools based on them. Finally, both are Java frameworks which facilitates their integration.

As further discussed in Chapter \ref{chap:background}, other tools were studied but JADE and Repast were the fittest for this thesis. Most MABS frameworks do not implement any interaction standards. Furthermore, JADE is a very rich framework and it is not the goal of this thesis to emulate all its features using Repast. With that said, the following guidelines were defined to achieve this thesis goals.

\begin{enumerate}
	\item To replicate JADE support for FIPA standards for agent interaction and management; these standards are better explained in Chapter \ref{chap:background}.
	\item To keep the API simple and fast, support a subset of commonly used JADE features; non fundamental features for local simulations, such as JADE's networking infrastructure, should be left out.
	\item Even though Repast is the featured framework for simulation development, the API should be sufficiently generic to allow support for new platforms to be added in future enhancements.
	\item Programmers should not need to introduce significant changes to their original applications in order to convert code created with the API; the conversion tool should be capable of converting the code \emph{as-is} and generate working models, which preserve the functionality of the original code.
\end{enumerate}


% To enable interaction standards in simulation frameworks;
% To develop an API that replicates the essential JADE features, while abdicating of JADE's networking   infrastructure and more complex internal features in order to gain in simulation performance;
% To develop an automatic code conversion tool (CCT) that uses this API to transform Repast-based simulations into JADE MAS;
% To add the possibility to convert some JADE MAS into simulations, based on the API and using the code conversion tool;
% To validate generated code by confirming that the execution of the simulation and the generated MAS produce identical behaviour;
% To make the API generic enough, allowing for future extension to support multiple simulation tools

\section{Contents}

This thesis documents the development of a tool that converts a MABS created in Repast into MAS that uses JADE. Conversely, it should allow the conversion of a JADE MAS into a Repast MABS as well. This tool is useful in the context of development of a MAS whose development started as a MABS or when the need to create a simulation arises during development. JADE and Repast were chosen for this thesis not only for their quality but also for their widespread use, available source code and documentation.

Chapter \ref{chap:background} starts by surveying tools whose goal is to produce MABS. Three frameworks were selected for a more detailed study because their approach is the most relevant to the goals of this thesis. They propose solutions based on enhancing JADE to enable simulations capabilities in it. The rest of the chapter is dedicated to comparing JADE and Repast's features and to include an introduction to some concepts regarding FIPA specifications.

Chapter \ref{chap:solution} provides a conceptual definition of SAJaS and MasSim2Dev, the developed tools, including an overview of their features and usage scenarios. This chapter also describes how FIPA Specifications are present in the API.

Chapter \ref{chap:architecture} gives a detailed description of the software architecture, including a description of how agents execute internally, in the API. This chapter is concluded with a discussion on the perspectives for extending the tools, referring to the discussion of future work in the conclusions.

Chapter \ref{chap:validation} presents scenarios used to validate the system. They were subjected to code conversion to verify that the correct execution of the code had been preserved. Their performance was also subject to analysis.

This thesis is concluded with a description of suggested future work and some final notes and conclusions.