%!TEX root = ../thesis.tex
\chapter{Automatic Code Conversion based on SAJaS}
\label{chap:solution}

The modelling of MAS can be accomplished through the use of a wide range of available platforms, as explained before. In certain cases, in order to benefit from different features, modelling the same system in multiple development environments may be necessary.

The developed solution consists in an integrated system comprised of two main components:

\begin{enumerate}
  \item The \textbf{Simple API for JADE-based Simulations (SAJaS)}, which provides a set of features present in JADE; those features were reimplemented from scratch in an attempt to simplify their internal complexity, preserving JADE-like external execution;
  \item The \textbf{Code Conversion Tool} in the form of an Eclipse plug-in that is capable of mapping JADE and SAJaS features and convert applications developed using one into applications based on the other, automatically.
\end{enumerate}

This chapter explains each of these components with detail, including a section on FIPA, and follows with a presenting with the expected use cases of the integrated system.

\section{Overview}
%In this section, explain the integrated solution of SAJaS + plug-in, how they work, what they use in terms of dependencies, etc.

SAJaS is an API meant to be used with simulation frameworks to enable JADE-based features in them, including agent interaction protocols and agent management services. The API also uses JADE's concept of behaviours which encapsulate most of agents' actions.

SAJaS was initially created to be used with Repast SImphony. However, it was developed in a way that allows its integration with other simulation tools. The interface between Repast and SAJaS is made in a single point, the Scheduler, which is a Repast-specific structure. BY creating a different implementation of the scheduler, it is possible to use other simulation frameworks.

\subsection{FIPA Specifications}

On of the main features of \apiname{} is the fact that it enables the development of FIPA-compliant MABS.
% Brief intro
As previously mentioned, \apiname{} closely follows JADE's architecture.
% The DF Agent Description in the api
It implements the DF Agent Description used to filter DF searches. Its focus is the development of local simulations (as opposed to distributed). As such, a single DF exists in each simulation. \apiname{}'s implementation of the MTS is simplified as well due to the absence of a distributed infrastructure considering that agent address resolution is unnecessary. The AMS contains a mapping of all AIDs to their respective agent objects, which is used by the MTS when delivering messages.
% The ACL message in the api
\apiname{}'s implementation of the ACL Message is slightly simpler than the one found in JADE, focusing on the most commonly used elements (the ones highlighted in Table \ref{tab:fipaACLMessage} from Chapter \ref{chap:background}).

% The interaction protocols in the api
The initial focus was the incorporation of \gls{FIPA} Interaction Protocols in Repast, but it grew to include the whole agent management infrastructure. The most common protocols were selected, following JADE's implementation, to be included in \apiname{}.
% The specific protocols
The implemented protocols were the ``request-like'' Achieve Rational Effect (AchieveRE) protocol, the Contract Net protocol and the Single Session Contract Net - including the Responder Dispatcher behaviour, which uses it. The AchieveRE encompasses multiple \gls{FIPA} protocols, namely Request, Query, Request-When, Recruiting and Brokering protocols, as defined in JADE's documentation. Chapter \ref{chap:architecture} explains with greater detail how agents execute in SAJaS and how they use behaviours and protocols.

\section{Usage Scenarios}
%In this section, show how the whole setup is used with detailed use cases
This section is meant to describe both the scenarios where this system is expected to be useful, as well as the actual possible use cases.

One possible scenario is when a JADE developer created a JADE MAS and desires to perform some tests and simulations in a local and controlled environment. The developer can use this tool to convert the MAS into a MABS. Eventually, the application can be converted back is changes were made while in the simulation format.

A second possible scenario could be that of a developer that intends to create a MABS with the goal of later creating a MAS out of it. This could be a Repast developer who desires to create more complex agent simulations or a JADE developer that wants to create Repast simulations using familiar JADE-like tools.

A third scenario is when a developer simply wants to create a complex agent-based, FIPA-compliant simulation. In this case, there is no need for a code conversion tool, but SAJaS can be used as a standalone library.

\subsection{Use Cases}

